%% ****** Start of file apstemplate.tex ****** %
%%
%%
%%   This file is part of the APS files in the REVTeX 4.2 distribution.
%%   Version 4.2a of REVTeX, January, 2015
%%
%%
%%   Copyright (c) 2015 The American Physical Society.
%%
%%   See the REVTeX 4 README file for restrictions and more information.
%%
%
% This is a template for producing manuscripts for use with REVTEX 4.2
% Copy this file to another name and then work on that file.
% That way, you always have this original template file to use.
%
% Group addresses by affiliation; use superscriptaddress for long
% author lists, or if there are many overlapping affiliations.
% For Phys. Rev. appearance, change preprint to twocolumn.
% Choose pra, prb, prc, prd, pre, prl, prstab, prstper, or rmp for journal
%  Add 'draft' option to mark overfull boxes with black boxes
%  Add 'showkeys' option to make keywords appear
\documentclass[aps,prl,twocolumn,groupedaddress]{revtex4-2}

% You should use BibTeX and apsrev.bst for references
% Choosing a journal automatically selects the correct APS
% BibTeX style file (bst file), so only uncomment the line
% below if necessary.
%\bibliographystyle{apsrev4-2}

\usepackage{graphicx}
\usepackage{epstopdf}
\usepackage{amssymb}
\usepackage{amsmath}
\usepackage{color}
\usepackage{hyperref}
\usepackage{tikz}
\usepackage{mathrsfs}
\usepackage{wasysym}
\usepackage{tikz-cd}
\usepackage{adjustbox}
\usepackage{booktabs}
\usepackage{orcidlink}


% https://tex.stackexchange.com/questions/269553/counter-too-large-error-with-item
\usepackage{alphalph}
%\renewcommand\thesubsectiondis{\AlphAlph{\value{subsection}}.}% for the headings in the text
\renewcommand%\thesubsection{\mbox{\thesection-\AlphAlph{\value{subsection}}}}% for the ToC, for example   \renewcommand
\thesubsection{\mbox{\AlphAlph{\value{subsection}}}}% for the ToC, for example   

% so sections, subsections, etc. become numerated.
\setcounter{secnumdepth}{3}

\DeclareMathOperator*{\argmax}{arg\,max}
\DeclareMathOperator*{\argmin}{arg\,min}
\newcommand{\avrg}[1]{\left\langle #1 \right\rangle}
\newcommand{\nelta}{\bar{\delta}}
\newcommand{\krn}{\mathrm{krn}}
\newcommand{\rng}{\mathrm{rng}}
\newcommand{\rnk}{\mathrm{rnk}}
\newcommand{\nll}{\mathrm{nll}}
\newcommand{\grad}{\mathrm{grad}}
\newcommand{\curl}{\mathrm{curl}}
\newcommand{\dive}{\mathrm{div}}
\newcommand{\bra}[1]{\left\langle #1\right|}
\newcommand{\ket}[1]{\left| #1 \right\rangle}
\newcommand{\sbra}[1]{\langle #1|}
\newcommand{\sket}[1]{| #1 \rangle}
\newcommand{\bek}[3]{\left\langle #1 \right| #2 \left| #3 \right\rangle}
\newcommand{\sbek}[3]{\langle #1 | #2 | #3 \rangle}
\newcommand{\braket}[2]{\left\langle #1 \middle| #2 \right\rangle}
\newcommand{\ketbra}[2]{\left| #1 \middle\rangle \middle\langle #2  \right|}
\newcommand{\sbraket}[2]{\langle #1 | #2 \rangle}
\newcommand{\sketbra}[2]{| #1 \rangle  \langle #2 |}
\newcommand{\norm}[1]{\left\lVert#1\right\rVert}
\newcommand{\snorm}[1]{\lVert#1\rVert}
\newcommand{\bvec}[1]{\boldsymbol{\mathsf{#1}}}
\newcommand{\bcov}[1]{\boldsymbol{#1}}
\newcommand{\bdua}[1]{\boldsymbol{\check{#1}}}
\newcommand{\bdov}[1]{\breve{#1}}
\newcommand{\bten}[1]{\boldsymbol{\mathfrak{#1}}}
\newcommand{\forany}{\tilde{\forall}}
\newcommand{\qed}{$\overset{\circ}{.}\;$}
\newcommand{\bigo}{\mathcal{O}}
\newcommand\scalemath[2]{\scalebox{#1}{\mbox{\ensuremath{\displaystyle #2}}}}

\newcommand{\QQ}{\mathbb{Q}}
\newcommand{\RR}{\mathbb{R}}
\newcommand{\CC}{\mathbb{C}}
\newcommand{\KK}{\mathbb{K}}
\newcommand{\FF}{\mathbb{F}}

\newcommand{\smallo}{o}

\begin{document}

% Use the \preprint command to place your local institutional report
% number in the upper righthand corner of the title page in preprint mode.
% Multiple \preprint commands are allowed.
% Use the 'preprintnumbers' class option to override journal defaults
% to display numbers if necessary
%\preprint{}

%Title of paper
\title{Analysis of the inference of ratings and rankings on Higher Order Networks with complex topologies}

% repeat the \author .. \affiliation  etc. as needed
% \email, \thanks, \homepage, \altaffiliation all apply to the current
% author. Explanatory text should go in the []'s, actual e-mail
% address or url should go in the {}'s for \email and \homepage.
% Please use the appropriate macro foreach each type of information

\author{Juan I. Perotti\,\orcidlink{https://orcid.org/0000-0001-7424-9552}}
\email[]{juan.perotti@unc.edu.ar}
\affiliation{Instituto de F\'isica Enrique Gaviola (IFEG-CONICET),\\
Facultad de Matem\'atica, Astronom\'ia, F\'isica y Computaci\'on,\\
Universidad Nacional de C\'ordoba, Ciudad Universitaria, 5000 C\'ordoba, Argentina}

%Collaboration name if desired (requires use of superscriptaddress
%option in \documentclass). \noaffiliation is required (may also be
%used with the \author command).
%\collaboration can be followed by \email, \homepage, \thanks as well.
%\collaboration{}
%\noaffiliation

\date{\today}

\begin{abstract}
Bla bla...
\end{abstract}

% insert suggested keywords - APS authors don't need to do this
%\keywords{}

%\maketitle must follow title, authors, abstract, and keywords
\maketitle

\section{Introduction}

See~\cite{perotti2025analysis,chacoma2025data}.

\section{Theory}
\label{sec:theory}

\subsection{Least-squares inference of team ratings and rankings}

Consider a set of $N=n+1$ teams labeled $\{0,1,\dots,n\}$.
For each distinct pair of teams $i$ and $j$, assume a pairwise antisymmetric comparison score $f_{ij}=-f_{ji}$ is available.
Informally, $f_{ij}>0$ indicates that team $i$ is rated higher than team $j$, whereas $f_{ij}<0$ suggests the opposite.

A natural approach to inferring consistent ratings is to assign scores $w_i$ ($i=0,1,\dots,n$) that minimize the total squared discrepancy
\begin{equation}
\label{eq:1}
\sum_{i<j} \bigl(f_{ij}-(w_i-w_j)\bigr)^2.
\end{equation}
Once the team ratings $w_0,w_1,\dots,w_n$ have been obtained, the ranking $r_0,r_1,\dots,r_n$ follows from the ordering condition $r_i<r_j$ whenever $w_i>w_j$.
This least-squares formulation defines a well-posed optimization problem that admits an elegant reformulation in terms of cochains over simplicial complexes.
The resulting framework for rating and ranking inference is known as \emph{HodgeRank}~\cite{jiang2011statistical, perotti2025analysis}.

\subsection{Discrete exterior calculus on simplicial complexes}

To review the HodgeRank method, we recall the basic elements of discrete exterior calculus on simplicial complexes.  
Let $V=\{0,1,\dots,n\}$ denote a set of vertices, and define a simplicial complex as a collection
\[
K = K_0 \cup K_1 \cup \dots \cup K_n,
\]
where
\[
K_k
\subseteq
\bigl\{
\{i_0,\dots,i_k\}
:
0\leq i_0<\dots<i_k\leq n
\bigr\}
\]
is the set of $k$-simplices, i.e., unordered subsets $\{i_0,\dots,i_k\}$ of vertices satisfying the closure condition: if $s\in K$ and $r\subset s$ with $r\neq \emptyset$, then $r\in K$.

The set $K$ of simplices is associated with a vector space $C$ of \emph{chains}, with canonical basis $\{e_s:s\in K\}$.
We equip $C$ with the inner product
\[
e_r \cdot e_s = \delta_{rs}, \qquad r,s \in K.
\]
With Einstein summation convention, any chain $c\in C$ can be written as $c=c^s e_s$ with coefficients $c^s\in \mathbb{R}$.
The chain space decomposes as
\[
C = C_0 \oplus C_1 \oplus C_2,
\]
where $C_k$ is spanned by $\{e_s:s\in K_k\}$.

The dual space $C^*$ of \emph{cochains} is spanned by $\{e^s:s\in K\}$, where $e^r(e_s)=\delta^r{}_s$.
Any $f\in C^*$ decomposes as $f=f_s e^s$ with coefficients $f_s\in \mathbb{R}$, and
\[
C^* = C_0^* \oplus C_1^* \oplus C_2^*.
\]

By Riesz’s theorem, every cochain $f\in C^*$ corresponds uniquely to a chain $f^\flat \in C$ such that $f(c) = f^\flat \cdot c$ for all $c\in C$.
This induces an inner product on $C^*$:
\[
f\cdot g = f^\flat \cdot g^\flat.
\]
The map $f\mapsto f^\flat$ is an isomorphism, with inverse denoted $c\mapsto c^\sharp$.

\medskip

\noindent\textbf{Orientation and boundary.}  
An orientation is fixed by requiring
\[
e_{i_0\dots i_k} = (-1)^p e_{j_0\dots j_k},
\]
for each simplex $s=\{i_0,\dots,i_k\}\in K$, where $p$ is the parity of the permutation mapping $(i_0,\dots,i_k)$ to $(j_0,\dots,j_k)$.

The boundary operator $\partial:C\to C$ is defined on basis elements as
\[
\partial e_{i_0\dots i_k}
=
\sum_{j=0}^k (-1)^j e_{i_0\dots i_{j-1}i_{j+1}\dots i_k}.
\]
Thus $\partial(C_k)\subseteq C_{k-1}$, with $C_{-1}=\{0\}$, yielding the decomposition
\[
\partial = \partial_0 \oplus \partial_1 \oplus \partial_2,
\qquad \partial_k(C_k)\subseteq C_{k-1}, \ \partial_0=0.
\]

\medskip

\noindent\textbf{Coboundary.}  
The coboundary operator $d:C^*\to C^*$ is defined by
\[
df(c) = f(\partial c), \qquad f\in C^*, \ c\in C.
\]
Hence $d(C_k^*) \subseteq C_{k+1}^*$, with $C_3^*=\{0\}$, and
\[
d = d_0 \oplus d_1 \oplus d_2, \qquad d_2=0.
\]

\medskip

\noindent\textbf{Adjoint operators.}  
The adjoints $\partial^*:C\to C$ and $d^*:C^*\to C^*$ are defined by
\[
\partial^*c \cdot q = c\cdot \partial q, 
\qquad 
d^*f \cdot g = f\cdot dg,
\]
for all $c,q\in C$ and $f,g\in C^*$.
They decompose as
\[
\partial^* = \partial_0^* \oplus \partial_1^* \oplus \partial_2^*, 
\qquad 
d^* = d_0^* \oplus d_1^* \oplus d_2^*.
\]

\noindent\textbf{Hodge Laplacians.}  
The boundary and coboundary operators define the Hodge Laplacians
\begin{eqnarray}
\mathcal{L} 
&=&
(\partial+\partial^*)^2 
=
\partial \partial^* + \partial^* \partial : C\to C,
\mbox{and}
\notag
\\
L &=& (d+d^*)^2 = dd^* + d^*d : C^*\to C^*,
\notag
\end{eqnarray}
with $\partial^2=(\partial^*)^2=d^2=(d^*)^2=0$.
These decompose as
\[
\mathcal{L} = \mathcal{L}_0 \oplus \mathcal{L}_1 \oplus \mathcal{L}_2, 
\qquad 
L = L_0 \oplus L_1 \oplus L_2,
\]
with
\[
\mathcal{L}_k = \mathcal{L}_k^{\downarrow}\oplus \mathcal{L}_k^{\uparrow}, 
\qquad 
L_k = L_k^{\downarrow}\oplus L_k^{\uparrow},
\]
\[
\mathcal{L}_k^{\downarrow} = \partial_k^*\partial_k, \quad 
\mathcal{L}_k^{\uparrow} = \partial_{k+1}\partial_{k+1}^*, 
\]
\[
L_k^{\uparrow} = d_k^*d_k, \quad 
L_k^{\downarrow} = d_{k-1}d_{k-1}^*.
\]

\noindent\textbf{Hodge decomposition.}  
Every $k$-cochain $f$ admits the Hodge decomposition
\[
f = g \oplus h \oplus s,
\]
where $g\in \operatorname{im} d_{k-1}$ (gradient), $h\in \ker L_k$ (harmonic), and $s\in \operatorname{im} d_k^*$ (solenoidal).
Equivalently, there exist $w\in C_{k-1}^*$ and $u\in C_{k+1}^*$ such that
\[
g = d_{k-1}w, \qquad s = d_k^* u,
\]
with $w$ and $u$ determined by
\begin{align}
\label{eq:2}
d_k f &= L_{k+1}^{\downarrow} u, \\
\label{eq:3}
d_{k-1}^* f &= L_{k-1}^{\uparrow} w,
\end{align}
while $h=f-g-s$.

%\subsection{HodgeRank}

The least-squares problem~\eqref{eq:1} admits a reformulation in the language of discrete calculus:
\begin{equation}
\label{eq:4}
\lVert f - d_0(w) \rVert,
\end{equation}
where the scalar entries $f_{ij}$ of the $1$-cochain $f=f_{ij}e^{ij}$ encode the pairwise comparison scores, and the components $w_i$ of the $0$-cochain $w=w_ie^i$ represent the team ratings.
The minimizer of~\eqref{eq:4} coincides with the solution of Eq.~\eqref{eq:3} for $k=1$.

In this setting, the set of $0$-simplices $K_0$ corresponds to the teams $V$, the set of $1$-simplices $K_1$ to the pairwise comparisons $E$, and the set of $2$-simplices $K_2$ to the closed triples of comparisons:
\[
K_2 = \bigl\{ \{i,j,k\} : \{i,j\},\{j,k\},\{i,k\}\in E \bigr\}.
\]
For $k>2$, $K_k=\emptyset$.

\section{Methods}
\label{sec:methods}

The dataset covers five major soccer leagues: England, France, Germany, Italy, and Spain.
Each league consists of $N=20$ teams, except Germany, which has $N=18$.

In these competitions (e.g., Premier League, La Liga, Serie A, Bundesliga, Ligue 1), the final standings are determined by a points system based on match results throughout the season.
In the standard system, a win yields 3 points, a draw 1 point, and a loss 0 points.
Each team plays every other team twice (home and away) in a round-robin format. 
Over the season, the $i$-th team accumulates $\pi_i$ total points.
At the end of the season, teams are ranked in descending order of $\pi_i$.
If two or more teams tie on points, tiebreakers are applied, typically in the following order (with minor variations across leagues):  
goal difference (GD: goals scored minus goals conceded),  
goals scored (GF),  
head-to-head record,  
and, in rare cases, a playoff or extra match.  
Thus, the final table is determined primarily by total points, with tiebreakers resolving ties.

%\subsection{Data}
\label{sec:data}

For each pair $\{i,j\}$ of teams in a given league, several pairwise antisymmetric comparison scores $f_{ij}$ are defined:
\begin{enumerate}
\item \textbf{crossings:} ...
\item \textbf{counterattacks:} ...
\item \textbf{pressure loss:} ...
\item \textbf{build-up time:} ...
\item \textbf{direct play:} ...
\item \textbf{pressure points:} ...
\item \textbf{shots:} ...
\item \textbf{flow rate:} ...
\item \textbf{maintenance time:} ...
\item \textbf{middle-zone time:} ...
\item \textbf{style:} the Singular Value Decomposition (SVD) of previous metrics is used to define higher-level descriptors called \emph{styles}, so $f_{ij}=...$~\cite{chacoma2025data}.
\item \textbf{rank based ground-truth:} $f_{ij}=i-j$.
\item \textbf{point based ground-truth:} $f_{ij}=\pi_j-\pi_i$.
\end{enumerate}

\subsection{HodgeRank inference via metrics}

Given a metric, the coefficients
\[
f_{ij}
=
-
\frac{1}{2}(m_{ij}+m_{ji})
\]
are defined for each $i<j$.
The minus sign is imposed to identify ratings as potentials~\cite{perotti2025analysis}.
To test the HodgeRank inference method, two types of ground-truth coefficients are also introduced:  
(i) \emph{rank-based} coefficients $f_{ij}=i-j$, and  
(ii) \emph{points-based} coefficients $f_{ij}=\pi_j-\pi_i$.

In all cases, the underlying structure is a fully connected graph with $N$ nodes, promoted to a simplicial complex composed of $0$-simplices labeled by $i$, $1$-simplices labeled by $ij$ with $i<j$, and $2$-simplices labeled by $ijk$ with $i<j<k$.
Correspondingly, the coefficients $f_{ij}$ form the components of a $1$-cochain $f$, where the antisymmetry $f_{ji}=-f_{ij}$ is imposed by definition.
There are no $k$-simplices for $k>2$.

\section{
Results
\label{sec:results}
}

The following experiments are carried on.
\begin{enumerate}
\item 
Compute ratings and rankings from the full weighted network of scores $\phi_{ij}$.

\item 
Compute ratings and rankings from sparsified versiond of the weighted networks of schores $\phi_{ij}$, by randomly removing a varying fraction of links.

\item 
Compute ratings and rankings from sparsified versiond of the weighted networks, by removing a varying fraction of the weighted links of the weighted matrix of scores $\phi_{ij}$ sorted in decreasing order of absolute weight.

\item 
Compute ratings defining the 1-cochain coordiantes $f'_{ij}=\phi_j-\phi_i$, where the strength $\phi_i = \sum_j w_{ij}$ defines a local potential for the $i$-th team.
\end{enumerate}

\begin{figure*}
\includegraphics*[scale=.4]{assets/fig1.pdf}
\includegraphics*[scale=.4]{assets/fig1.pdf}
%\put(-483,140){\bf a)}
%\put(-240,140){\bf b)}
\\
\includegraphics*[scale=.4]{assets/fig1.pdf}
\includegraphics*[scale=.4]{assets/fig1.pdf}
%\put(-483,140){\bf c)}
%\put(-240,140){\bf d)}
\caption{
\label{fig:1}
(color online).
}
\end{figure*}

\section{
Conclusions
\label{sec:conclusions}
}

\begin{acknowledgments}
The author acknowledges partial support from CONICET under grant PIP2021-2023 number 11220200101100, and from SeCyT (Universidad Nacional de Córdoba, Argentina). Special thanks go to O.V. Billoni for a careful reading of the article, and to the Centro de Cómputo de Alto Desempeño (CCAD) at the Universidad Nacional de Córdoba (UNC), \url{http://ccad.unc.edu.ar/}, which is part of SNCAD – MinCyT, Argentina, for providing computational resources.
\end{acknowledgments}

% Create the reference section using BibTeX:
\bibliography{ref}

% Specify following sections are appendices. Use \appendix* if there
% only one appendix.

%\appendix*
%\onecolumngrid

%\section{
%Combinatorial differential topology, geometry and calculus
%\label{appA}
%}

\end{document}
%
% ****** End of file apstemplate.tex ******


