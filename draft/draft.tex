%% ****** Start of file apstemplate.tex ****** %
%%
%%
%%   This file is part of the APS files in the REVTeX 4.2 distribution.
%%   Version 4.2a of REVTeX, January, 2015
%%
%%
%%   Copyright (c) 2015 The American Physical Society.
%%
%%   See the REVTeX 4 README file for restrictions and more information.
%%
%
% This is a template for producing manuscripts for use with REVTEX 4.2
% Copy this file to another name and then work on that file.
% That way, you always have this original template file to use.
%
% Group addresses by affiliation; use superscriptaddress for long
% author lists, or if there are many overlapping affiliations.
% For Phys. Rev. appearance, change preprint to twocolumn.
% Choose pra, prb, prc, prd, pre, prl, prstab, prstper, or rmp for journal
%  Add 'draft' option to mark overfull boxes with black boxes
%  Add 'showkeys' option to make keywords appear
\documentclass[aps,prl,twocolumn,groupedaddress]{revtex4-2}

% You should use BibTeX and apsrev.bst for references
% Choosing a journal automatically selects the correct APS
% BibTeX style file (bst file), so only uncomment the line
% below if necessary.
%\bibliographystyle{apsrev4-2}

\usepackage{graphicx}
\usepackage{epstopdf}
\usepackage{amssymb}
\usepackage{amsmath}
\usepackage{color}
\usepackage{hyperref}
\usepackage{tikz}
\usepackage{mathrsfs}
\usepackage{wasysym}
\usepackage{tikz-cd}
\usepackage{adjustbox}
\usepackage{booktabs}
\usepackage{orcidlink}


% https://tex.stackexchange.com/questions/269553/counter-too-large-error-with-item
\usepackage{alphalph}
%\renewcommand\thesubsectiondis{\AlphAlph{\value{subsection}}.}% for the headings in the text
\renewcommand%\thesubsection{\mbox{\thesection-\AlphAlph{\value{subsection}}}}% for the ToC, for example   \renewcommand
\thesubsection{\mbox{\AlphAlph{\value{subsection}}}}% for the ToC, for example   

% so sections, subsections, etc. become numerated.
\setcounter{secnumdepth}{3}

\DeclareMathOperator*{\argmax}{arg\,max}
\DeclareMathOperator*{\argmin}{arg\,min}
\newcommand{\avrg}[1]{\left\langle #1 \right\rangle}
\newcommand{\nelta}{\bar{\delta}}
\newcommand{\krn}{\mathrm{krn}}
\newcommand{\rng}{\mathrm{rng}}
\newcommand{\rnk}{\mathrm{rnk}}
\newcommand{\nll}{\mathrm{nll}}
\newcommand{\grad}{\mathrm{grad}}
\newcommand{\curl}{\mathrm{curl}}
\newcommand{\dive}{\mathrm{div}}
\newcommand{\bra}[1]{\left\langle #1\right|}
\newcommand{\ket}[1]{\left| #1 \right\rangle}
\newcommand{\sbra}[1]{\langle #1|}
\newcommand{\sket}[1]{| #1 \rangle}
\newcommand{\bek}[3]{\left\langle #1 \right| #2 \left| #3 \right\rangle}
\newcommand{\sbek}[3]{\langle #1 | #2 | #3 \rangle}
\newcommand{\braket}[2]{\left\langle #1 \middle| #2 \right\rangle}
\newcommand{\ketbra}[2]{\left| #1 \middle\rangle \middle\langle #2  \right|}
\newcommand{\sbraket}[2]{\langle #1 | #2 \rangle}
\newcommand{\sketbra}[2]{| #1 \rangle  \langle #2 |}
\newcommand{\norm}[1]{\left\lVert#1\right\rVert}
\newcommand{\snorm}[1]{\lVert#1\rVert}
\newcommand{\bvec}[1]{\boldsymbol{\mathsf{#1}}}
\newcommand{\bcov}[1]{\boldsymbol{#1}}
\newcommand{\bdua}[1]{\boldsymbol{\check{#1}}}
\newcommand{\bdov}[1]{\breve{#1}}
\newcommand{\bten}[1]{\boldsymbol{\mathfrak{#1}}}
\newcommand{\forany}{\tilde{\forall}}
\newcommand{\qed}{$\overset{\circ}{.}\;$}
\newcommand{\bigo}{\mathcal{O}}
\newcommand\scalemath[2]{\scalebox{#1}{\mbox{\ensuremath{\displaystyle #2}}}}

\newcommand{\QQ}{\mathbb{Q}}
\newcommand{\RR}{\mathbb{R}}
\newcommand{\CC}{\mathbb{C}}
\newcommand{\KK}{\mathbb{K}}
\newcommand{\FF}{\mathbb{F}}

\newcommand{\smallo}{o}

\begin{document}

% Use the \preprint command to place your local institutional report
% number in the upper righthand corner of the title page in preprint mode.
% Multiple \preprint commands are allowed.
% Use the 'preprintnumbers' class option to override journal defaults
% to display numbers if necessary
%\preprint{}

%Title of paper
\title{Analysis of the inference of ratings and rankings on Higher Order Networks with complex topologies}

% repeat the \author .. \affiliation  etc. as needed
% \email, \thanks, \homepage, \altaffiliation all apply to the current
% author. Explanatory text should go in the []'s, actual e-mail
% address or url should go in the {}'s for \email and \homepage.
% Please use the appropriate macro foreach each type of information

\author{Juan I. Perotti\,\orcidlink{https://orcid.org/0000-0001-7424-9552}}
\email[]{juan.perotti@unc.edu.ar}
\affiliation{Instituto de F\'isica Enrique Gaviola (IFEG-CONICET),\\
Facultad de Matem\'atica, Astronom\'ia, F\'isica y Computaci\'on,\\
Universidad Nacional de C\'ordoba, Ciudad Universitaria, 5000 C\'ordoba, Argentina}

%Collaboration name if desired (requires use of superscriptaddress
%option in \documentclass). \noaffiliation is required (may also be
%used with the \author command).
%\collaboration can be followed by \email, \homepage, \thanks as well.
%\collaboration{}
%\noaffiliation

\date{\today}

\begin{abstract}
Bla bla...
\end{abstract}

% insert suggested keywords - APS authors don't need to do this
%\keywords{}

%\maketitle must follow title, authors, abstract, and keywords
\maketitle

\section{Introduction}

See~\cite{perotti2025analysis,chacoma2025data}.

\section{
Theory
\label{sec:theory}
}

Bla bla...

\section{
Methods
\label{sec:methods}
}

\subsection{
Data
\label{sec:data}
}

The dataset covers five soccer leagues: England, France, Germany, Italy, and Spain.
Each league consists of $N=20$ teams, except for Germany, which has $N=18$.
Every pair of teams $i,j$ plays two matches: one with $i$ as the home team and $j$ as the away team, and in the other match it is the other way around.
The teams are indexed or ranked according to their position in that year’s final standings.
The standings is characterized by some tournament points $\pi_i$ for each team $i$ in each league.
Therefore, the condition $i<j$ holds if and only if $\pi_i>\pi_j$ for any two teams $i$ and $j$ in a given league.

For each pair of teams $i$ and $j$ in a given league, several pairwise metrics $m_{ij}\in \mathbb{R}$ are provided.
Each metric takes a positive value if it presumably favors team $i$ over team $j$, and a negative value otherwise.
The different types of empirically determined metrics being considered are:
\begin{enumerate}
\item {\bf crossings:}
\item {\bf counterattacks:}
\item {\bf pressure loss:}
\item {\bf build-up time:}
\item {\bf direct play:}
\item {\bf pressure points:}
\item {\bf shots:}
\item {\bf flow rate:}
\item {\bf maintenance time:}
\item {\bf middle-zone time:} bla bla.
\end{enumerate}
Namely, $m_{ij}$ corresponds to $i$ being the home team and $j$ being the away team.
Additionally, metric-styles derived from the Singular Value Decomposition of previous metrics are also considered~\cite{chacoma2025data}.
Notice, $m_{ij}$ is generally independent of the metric $m_{ji}$ since they belong to different matches between $i$ and $j$.
As a consequence, $m_{ij}+m_{ji}$ is generally different from zero and is the net contribution between teams $i$ and $j$.

\subsection{Definition of 1-cochains}

With each metric, a 1-cochain is defined as
$$
f_{ij}
=
\frac{1}{2}(m_{ij}+m_{ji})
$$
Moreover, in order to test the inference method, two types of ground-truth 1-cochains are also defined.
Namely, the rank-based 1-cochain $f_{ij}=i-j$ and the tournament points based 1-cochain $f_{ij}=\pi_j-\pi_i$.
Note, $f_{ij}=-f_{ji}$ holds by definition.
These 1-cochains $f$ define a fully connected network or graph since, as already mentioned, all pairs of teams play two matches against each other in each league.
Each of these fully connected graphs of $N=n+1$ nodes is promoted to a simplicial complex composed of 0-simplices labeled by $i$, 1-simplices labeled by $ij$ and 2-simplices labeled by $ijk$ where $i,j$ and $k$ run over all choices in $\{0,1,2,...,n\}$ such that $i<j$ or $i<j<k$, depending on the case.

From a score $\phi$, a corresponding cochain $f\in C_1^*$ of coordinates $f_{ij}=\phi_{ji}-\phi_{ij}$ is defined.
Also, the cochain $f'\in C_1^*$ of coordinates $f'_{ij}=\phi_j-\phi_i$ with team strengths $\phi_i=\sum_j \phi_{ij}$ is considered.

\section{
Results
\label{sec:results}
}

The following experiments are carried on.
\begin{enumerate}
\item 
Compute ratings and rankings from the full weighted network of scores $\phi_{ij}$.

\item 
Compute ratings and rankings from sparsified versiond of the weighted networks of schores $\phi_{ij}$, by randomly removing a varying fraction of links.

\item 
Compute ratings and rankings from sparsified versiond of the weighted networks, by removing a varying fraction of the weighted links of the weighted matrix of scores $\phi_{ij}$ sorted in decreasing order of absolute weight.

\item 
Compute ratings defining the 1-cochain coordiantes $f'_{ij}=\phi_j-\phi_i$, where the strength $\phi_i = \sum_j w_{ij}$ defines a local potential for the $i$-th team.
\end{enumerate}

\begin{figure*}
\includegraphics*[scale=.4]{assets/fig1.pdf}
\includegraphics*[scale=.4]{assets/fig1.pdf}
%\put(-483,140){\bf a)}
%\put(-240,140){\bf b)}
\\
\includegraphics*[scale=.4]{assets/fig1.pdf}
\includegraphics*[scale=.4]{assets/fig1.pdf}
%\put(-483,140){\bf c)}
%\put(-240,140){\bf d)}
\caption{
\label{fig:1}
(color online).
}
\end{figure*}

\section{
Conclusions
\label{sec:conclusions}
}

\begin{acknowledgments}
The author acknowledges partial support from CONICET under grant PIP2021-2023 number 11220200101100, and from SeCyT (Universidad Nacional de Córdoba, Argentina). Special thanks go to O.V. Billoni for a careful reading of the article, and to the Centro de Cómputo de Alto Desempeño (CCAD) at the Universidad Nacional de Córdoba (UNC), \url{http://ccad.unc.edu.ar/}, which is part of SNCAD – MinCyT, Argentina, for providing computational resources.
\end{acknowledgments}

% Create the reference section using BibTeX:
\bibliography{ref}

% Specify following sections are appendices. Use \appendix* if there
% only one appendix.

%\appendix*
%\onecolumngrid

%\section{
%Combinatorial differential topology, geometry and calculus
%\label{appA}
%}

\end{document}
%
% ****** End of file apstemplate.tex ******


